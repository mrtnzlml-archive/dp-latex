\chapter{Úvod}
\pagenumbering{arabic}

V~dnešní době existují tisíce různých zdrojů dat a tisíce meteorologických stanic od různých výrobců. Přesto není běžně rozšířený žádný společný komunikační protokol. Tato diplomová práce si klade za úkol zamyslet se nad touto skutečností, navrhnout řešení a tento nápad prakticky zrealizovat. Cílem je umožnit získávání dat z~meteostanic, jejich ukládání a vizualizace prostřednictvím webové aplikace. Na takový problém je třeba dívat se dostatečně široce. Proto je jako doplňkový úkol nutné obohatit získaná data z~meteorologických stanic o~obrazový záznam z~webových kamer.

Bylo by naivní snažit se vytvořit takovou aplikaci, která bude znát každou meteostanici. Jednak je to vzhledem k~jejich počtu nadlidský úkol a potom ne každá meteostanice dokáže data odesílat přes síť na případný server. Výsledná aplikace na základě zadání této diplomové práce vykrystalizovala v~poměrně zajímavý návrh. Základem je server, který vnějšímu světu nabízí GraphQL API a to je také jediný způsob, jak lze se serverem komunikovat. Toto API přijímá primárně informace o~fyzikálních veličinách s~tím, že nezáleží na jejich původu. Server tak tvoří jediný zdroj pravdy. Stále však existuje problém s~velkým množstvím různých meteostanic. Ten je vyřešen pomocí převodníků (tzv. konkretizačních uzlů), které mají za úkol porozumět jedné konkrétní meteostanici a odesílat její data přes API na server. Tento konkretizační uzel je doslova několik desítek řádek kódu.

Data přijatá prostřednictvím API jsou zobrazována ve webovém pro\-hlí\-že\-či. K~tomu je použit systém React komponent, který z~toho samého API dokáže data velmi elegantním způsobem získat. Ona elegance spočívá právě v~GraphQL, které poskytuje grafové API. Díky tomu je možné dotázat se serveru na konkrétní podmožinu dat namísto předem definované množiny dat.

Celou situaci příjemně komplikují webové kamery. Ty se totiž zcela vymykají způsobům, kterými komunikují meteostanice. Webová kamera poskytuje tok dat (video záznam), který nelze rozumným způsobem odesílat přes žádné API. Navíc tento tok dat není vhodný pro přímé zpracování v~přehrávači videa a je nutné nejprve provést konverzi formátu. Součástí práce je tedy také mikroslužba, která se o~toto předzpracování záznamu stará. Hlavní API server pak pouze informuje tuto službu s~žádostí o~nové zpracování streamu dat.

Výsledkem je překvapivě komplexní aplikace, která skutečně dokáže zpracovávat data z~libovolných meteostanic a dokáže přehrávat videozáznamy z~webových kamer v~prohlížeči. Největší síla je však v~GraphQL API, které z~hlediska uživatele sice není nijak vidět, ale z~hlediska celého návrhu tvoří hlavní pilíř řešící téměř jakýkoliv myslitelný problém. Důkazem tohoto tvrzení jsou například exporty dat. Exportování nebylo nikdy v~plánu vytvářet, ale díky GraphQL lze exportovat jakákoliv data dostupná prostřednictvím API.

Velký důraz při vypracovávání této diplomové práce byl kladen na moderní technologie a postupy. Byly použity poslední verze Nette Frameworku (PHP), Reactu (JS), PostgreSQL a dalších nástrojů a knihoven. Stejně tak byl kladen velký důraz na dobrý návrh architektury, která z~velké části vychází z~principů DDD a automaticky testovatelného kódu \cite{ddd}.
