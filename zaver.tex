\chapter{Hodnocení a závěr}

Prvotním cílem práce bylo naléz způsob, jak získávat data z~různých meteorologických stanic a následně je zobrazovat ve vytvořené webové aplikaci. Aby měl uživatel aplikace také vizuální představu o~aktuálním počasí, data bylo nutné doplnit o~záznam z~kamery. Již na samotném začátku práce se ukázalo, že získávat data ze starších meteostanic je velice komplikované. Takové meteostanice totiž nejsou nijak připojeny k~síti a jediný způsob, jak komunikují, je přes sériovou linku přímo s~připojeným počítačem, na kterém musí běžet software od výrobce stanice. Rychle se tak ukázalo, že není možné napsat aplikaci, která by znala konkrétní implementační detaily jednotlivých zařízení. Tento poznatek následně ovlivnil celé výsledné řešení.

Výsledkem této diplomové práce je hlavní server, který nemá žádnou znalost o~jednotlivých meteostanicích. Rozumí však velmi dobře fyzikálním veličinám a s~okolním světem komunikuje pouze prostřednictvím GraphQL API. Na toto API jsou z~jedné strany napojeny konkretizační členy, které obsahují jednotlivé implementační detaily meteorologických stanic. Z~druhé strany API je napojena webová aplikace, která je napsána v~Reactu a kde může uživatel celý systém ovládat. API tak slouží pro obousměrnou komunikaci a jeho velkou přednostní je grafový charakter. Data jsou k~dispozici v~podobě grafu, takže je možné získat pouze konkrétní podmnožinu tohoto grafu a ušetřit tak přenášená data.

Kromě serveru, který poskytuje GraphQL API, byla vytvořena také aplikace pro streamování videa z~webových kamer. Tyto kamery zpravidla neposkytují formát, který je vhodný pro přímé zpracování ve webovém prohlížeči a je proto nutné provádět tzv. \uv{near real-time} konverzi obrazu. Streamovací server je postaven jako samostatná mikroslužba. Díky tomu, že není součástí serveru, který poskytuje GraphQL API, tak je možné tuto část aplikace nazávisle škálovat a dosahovat tak požadovaného výkonu.

Zadání práce bylo tedy nejen úspěšně vyřešeno, ale také rozšířeno o~implementaci GraphQL, které se stalo ústředním bodem celé práce. Výsledná aplikace je díky API a hlavně díky systému konkretizačních členů schopna pracovat s~jakoukoliv meteorologickou stanicí, která je na trhu.

Do budoucna by bylo vhodné vylepšit systém konkretizačních členů. Je běžné, že kvůli jedné meteostanici musí běžet celé dny zapnutý kancelářský počítač. Přitom důvod je pouze ten, že obsahuje software od výrobce, sériový port a připojení k~internetu. V~tom vidím velkou rezervu. Stačilo by připojit stanici k~jednoduchému převodníku, který by disponoval sériovým portem a uměl data odesílat do sítě. Velkou překážkou jsou však samotní výrobci meteorologických stanic. Často totiž není k~dispozici dostatečně podrobná technická dokumentace a porozumět všem náležitostem komunikace přes sériovou linku může být nepřekonatelný úkol. Naštěstí novější meteostanice umějí odesílat data přímo do sítě a tím tento problém zcela mizí.
